\documentclass[12pt]{article}
\usepackage[utf8]{inputenc}
\usepackage{graphicx}
\usepackage[colorlinks=true, linkcolor=black]{hyperref}
\usepackage{float} 
\usepackage{caption}
\usepackage{subcaption}
\usepackage{multicol}
\usepackage{comment}
\usepackage{booktabs} % For better-looking tables 
\usepackage{array} % For more control over table column formatting
\usepackage{colortbl} %Coloring table
\usepackage{xcolor}%Support hex values of color

\begin{document}

%%%%%%%%%%%%%%%%%%%%%%                Front page               %%%%%%%%%%%%%%%%%%%%%%%

\title{\textbf{Leadership strategies for women}}

\author{ Jyothy Das}
\maketitle

%%%%%%%%%%%%%%%%%     Table of Contents, List of Figures, and List of Tables   %%%%%%%%%%%%%%
\newpage
\tableofcontents
\newpage
\listoffigures
\newpage
\listoftables

%%%%%%%%%%%%%%%%%%%                                Introduction                       %%%%%%%%%%%%%%%%%%%%
\newpage
\section{Introduction}
\subsection{Women lead differently}
Skills acquired through this course are:\\
\textbf{Understanding Gender Differences:} Recognize the strengths of both male and female brains and how these differences impact leadership styles.\\
\textbf{Skill Development:} Learn methods to optimize time spent developing your skills and identify strategies to obtain useful feedback.\\
\textbf{Navigating Challenges:} Understand common challenges women in leadership face, such as gender bias, communication barriers, and imposter syndrome, and learn strategies to overcome them.\\
\textbf{Negotiation and Communication:} Discover the strengths women bring to negotiations and effective communication strategies, including body language and assertiveness.\\
\textbf{Inclusive Leadership:} Gain insights into creating an inclusive and collaborative culture, encouraging male allies, and leveraging both masculine and feminine leadership qualities.\\

\subsection{What organizations owe women}
\textbf{Systemic Bias:} The video acknowledges the unfairness and systemic bias that women face in the workplace, which is not their fault.\\
\textbf{Organizational Responsibility:} It emphasizes that organizational leaders and change makers need to address these biases and create inclusive environments.\\
\textbf{Inclusive Culture:} Companies should encourage women to be their true selves and set the stage for their success, rather than expecting them to change to fit a masculine company culture.\\
\textbf{Practical Steps:} Organizations should check their biases, ensure pay equity, and provide resources to support women and minorities, creating a culture that benefits all employees.\\

%%%%%%%%%%%%%%%%%%%%%%                     Gender Intelligence      %%%%%%%%%%%%%%%%%%%%
\newpage
\section{Gender Intelligence}
\subsection{We all have gender bias }
\textbf{Unconscious Bias:}  Both men and women have subconscious attitudes that can disadvantage women in the workplace, often associating men with careers and women with family.\\
\textbf{Confirmation Bias:}  This is the tendency to look for evidence that supports existing beliefs, which can reinforce and perpetuate biases.\\
\textbf{Reducing Bias:}  Regular self-reflection and adjusting behavior based on initial thoughts can help reduce unconscious bias and lead to fairer treatment of all individuals.

\subsection{You work like you played}
\textbf{Early Socialization:}  Boys and girls are socialized differently from an early age, which influences their behavior as leaders. Girls often learn cooperation and continuous improvement, while boys focus on competition and winning.\\
\textbf{Leadership Strengths:}  These socialization differences give women strengths in leadership, such as striving for continuous improvement, being conscious of the overall process, and taking responsibility for the team's performance.\\
\textbf{Team Dynamics:}  Women tend to create a team environment with less conflict and a stronger focus on ethical considerations, valuing cooperation and the process over just the end result.

\subsection{Same words different definitions}

\textbf{Miscommunication:}  Men and women can use the same words but understand them differently, leading to miscommunication.\\
\textbf{Clarifying Intentions:}  It's important to clarify the intention behind questions or statements to avoid misunderstandings. Both senders and receivers should ensure they understand each other's expectations.\\
\textbf{Communication Tendencies:}  Women often look for areas of agreement and bond through conversation, while men may focus on gaps and bond through tasks. Recognizing these tendencies can help improve communication.

\subsection{Listening to diagnose Vs Listening to problem solve}
\textbf{Listening Styles:}  Men tend to be action-oriented listeners, focusing on defining problems and finding solutions, while women are people-oriented listeners, connecting with the emotional undertones of conversations.\\
\textbf{Nonverbal Responses:}  Women often provide more verbal and nonverbal responses to show understanding, whereas men may nod to show agreement or interrupt for clarification.\\
\textbf{Improving Communication:}  Define the purpose of the conversation and clarify nonverbal cues to ensure mutual understanding and effective communication.
%%%%%%%%%%%%%%%%%%%%%                   Leadership strategies for women                    %%%%%%%%%%%%%%%%%%%
\section{  Leadership strategies for women}

%%%%%%%%%%%%%%%%%%%%                   Communication strategies for women                    %%%%%%%%%%%%%%%%%%%
\section{ Communication strategies for women}


\subsection{Key Areas of Study}
\subsubsection{Overview}

%%%%%%%%%%%%%%% 				Championing women's leadership			%%%%%%%%%%%%%%%%%%%%%
\section{Championing women's leadership}





%%%%%%%%%%%%%%%%%%%%%%   Table 1 
\begin{center}
\begin{tabular}{||c|c||}
\hline
\hline
\rowcolor[HTML]{89A8B2} Column 1 & Column 2 \\
\hline
\hline
\rowcolor[HTML]{F8FAFC} Data 1 & Data 2 \\ % Light gray
\hline

\rowcolor[HTML]{D8EFD3} Data 4 & Data 5 \\
\hline

\rowcolor[HTML]{F8FAFC} Data 6 & Data 7 \\ % Light blue
\hline
\hline
\end{tabular}
\end{center}
\bigskip % Add some vertical space

\subsubsection{}



\end{document}
