\documentclass[12pt]{article}
\usepackage[utf8]{inputenc}
\usepackage{graphicx}
\usepackage[colorlinks=true, linkcolor=black]{hyperref}
\usepackage{float} 
\usepackage{caption}
\usepackage{subcaption}
\usepackage{multicol}
\usepackage{comment}
\usepackage{booktabs} % For better-looking tables 
\usepackage{array} % For more control over table column formatting
\usepackage{colortbl} %Coloring table
\usepackage{xcolor}%Support hex values of color

\begin{document}

%%%%%%%%%%%%%%%%%%%%%%                Front page               %%%%%%%%%%%%%%%%%%%%%%%

\title{\textbf{Leadership strategies for women}}
\author{ Carolyn Goerner and Daisy Lovelace}
\maketitle

%%%%%%%%%%%%%%%%%      Table of Contents, List of Figures, and List of Tables       %%%%%%%%%%%%%%
\newpage
\tableofcontents
\newpage
%\listoffigures
%\newpage
%\listoftables

%%%%%%%%%%%%%%%%%%%%%%%%%%%%       Introduction         %%%%%%%%%%%%%%%%%%%%%%%%%%%%%
\newpage
\section{Introduction}

%%%%%%%
\subsection{Women lead differently}
Skills acquired through this course are:\\
\textbf{Understanding gender differences:} Recognize the strengths of both male and female brains and how these differences impact leadership styles.\\
\textbf{Skill development:} Learn methods to optimize time spent developing your skills and identify strategies to obtain useful feedback.\\
\textbf{Navigating challenges:} Understand common challenges women in leadership face, such as gender bias, communication barriers, and imposter syndrome, and learn strategies to overcome them.\\
\textbf{Negotiation and communication:} Discover the strengths women bring to negotiations and effective communication strategies, including body language and assertiveness.\\
\textbf{Inclusive leadership:} Gain insights into creating an inclusive and collaborative culture, encouraging male allies, and leveraging both masculine and feminine leadership qualities.\\

%%%%%%%
\subsection{What organizations owe women}
\textbf{Systemic bias:} The video acknowledges the unfairness and systemic bias that women face in the workplace, which is not their fault.\\
\textbf{Organizational Responsibility:} It emphasizes that organizational leaders and change makers need to address these biases and create inclusive environments.\\
\textbf{Inclusive culture:} Companies should encourage women to be their true selves and set the stage for their success, rather than expecting them to change to fit a masculine company culture.\\
\textbf{Practical steps:} Organizations should check their biases, ensure pay equity, and provide resources to support women and minorities, creating a culture that benefits all employees.\\

%%%%%%%%%%%%%%%%%%%%%%                     Gender Intelligence      %%%%%%%%%%%%%%%%%%%%
\newpage
\section{Gender intelligence}

%%%%%%%
\subsection{We all have gender bias }
\textbf{Unconscious bias:}  Both men and women have subconscious attitudes that can disadvantage women in the workplace, often associating men with careers and women with family.\\
\textbf{Confirmation bias:}  This is the tendency to look for evidence that supports existing beliefs, which can reinforce and perpetuate biases.\\
\textbf{Reducing bias:}  Regular self-reflection and adjusting behavior based on initial thoughts can help reduce unconscious bias and lead to fairer treatment of all individuals.

%%%%%%%
\subsection{You work like you played}
\textbf{Early socialization:}  Boys and girls are socialized differently from an early age, which influences their behavior as leaders. Girls often learn cooperation and continuous improvement, while boys focus on competition and winning.\\
\textbf{Leadership strengths:}  These socialization differences give women strengths in leadership, such as striving for continuous improvement, being conscious of the overall process, and taking responsibility for the team's performance.\\
\textbf{Team dynamics:}  Women tend to create a team environment with less conflict and a stronger focus on ethical considerations, valuing cooperation and the process over just the end result.

%%%%%%%
\subsection{Same words different definitions}
\textbf{Miscommunication:}  Men and women can use the same words but understand them differently, leading to miscommunication.\\
\textbf{Clarifying intentions:}  It's important to clarify the intention behind questions or statements to avoid misunderstandings. Both senders and receivers should ensure they understand each other's expectations.\\
\textbf{Communication tendencies:}  Women often look for areas of agreement and bond through conversation, while men may focus on gaps and bond through tasks. Recognizing these tendencies can help improve communication.

%%%%%%%
\subsection{Listening to diagnose Vs Listening to problem solve}
\textbf{Listening styles:}  Men tend to be action-oriented listeners, focusing on defining problems and finding solutions, while women are people-oriented listeners, connecting with the emotional undertones of conversations.\\
\textbf{Nonverbal responses:}  Women often provide more verbal and nonverbal responses to show understanding, whereas men may nod to show agreement or interrupt for clarification.\\
\textbf{Improving communication:}  Define the purpose of the conversation and clarify nonverbal cues to ensure mutual understanding and effective communication.

%%%%%%%
\subsection{Chapter Quiz}
\textbf{Q1.}  Jada is providing Mateo with updates on a social media marketing strategy. In order to ensure that words have the same meanings, Jada should make sure she clarifies her \_\_\_\_.\\
a)issues b)intentions c)questions d) agreements\\
\textbf{Ans: b}\\
Clarifying your intentions would give you the best chance in making sure that your words have the same meanings.\\
\textbf{Q2.} Which leadership advantage do women have based on socialization in their early life?
a) Strive for continuous improvement.\\
b) Get immunity from responsibility.\\
c) Focus on the end result.\\
d) Use conflict strategies.\\
\textbf{Ans: a}\\
\textbf{Q3.} Ethan is looking at some resumes and notices that a candidate has a Harvard degree. Perfect, he is going hire this person. This is an example of \_\_\_\_\_.\\
a) negativity bias \\
b) self-serving bias \\
c) confirmation bias\\
d) unconscious bias\\
\textbf{Ans: d}\\
In this case, the bias, also known as implicit associations, is a jump to judgment because the candidate is from Harvard.

%%%%%%%%%%%%%%%%%%%%%          Leadership strategies for women          %%%%%%%%%%%%%%%%%%%
\section{ Leadership strategies for women}

%%%%%%%
\subsection{Choosing roles that showcase your skills}
\textbf{Plan ahead:} Have a clear idea of the work you want to do so you can pursue the right opportunities and avoid tasks that don't interest you.\\
\textbf{Avoid grunt work:} Don't volunteer for leftover or mundane tasks, as it can send the wrong message about your enthusiasm and initiative.\\
\textbf{Communicate your preferences:} Politely decline office chores or grunt work and ensure your roles showcase your expertise, spreading the less desirable tasks among the team.

%%%%%%%
\subsection{Overcoming imposter syndrome}
\textbf{Recognize the feelings:} Identify when you're experiencing imposter syndrome and acknowledge it.\\
\textbf{Talk about it:} Share your feelings with a trusted friend or mentor to gain perspective and support.\\
\textbf{Reframe your thoughts:} Change negative thoughts into positive affirmations to approach challenges more positively.\\
\textbf{Collect positive experiences:}  Pay attention to and remember positive feedback and compliments.\\
\textbf{Consider professional help:}  If imposter syndrome is deeply rooted, seek professional guidance to manage it effectively.\\

%%%%%%%
\subsection{Managing anger and stress}
\textbf{Control Your Body:}  Use deep breaths to slow your heart rate and engage your body by sitting up straight and clenching/relaxing parts of your lower body to distract the amygdala.\\
\textbf{Define the Situation:}  Identify and name your emotions to push work to the rational prefrontal cortex. For example, think "I'm angry because..." or "I'm frustrated by...".\\
\textbf{Take a Break:} Use strategies like finishing a water bottle and excusing yourself to get a refill, or looking at yourself in a mirror to decrease anxiety and regain control.\\

These tips help you manage emotions effectively and maintain a rational mindset at work.\\

%%%%%%%
\subsection{Getting good feedback as a female leader}
\textbf{Ask specific questions:}  Instead of broad questions like "How can I improve?", ask about specific competencies to get actionable feedback.\\
\textbf{Seek strategies:}  Inquire about how to develop proficiency in specific areas, and ask your manager what they did to improve those skills.\\
\textbf{Develop relationships:}  Build connections with multiple leaders in your organization, as they can provide diverse perspectives and support your ambitions.\\
These strategies can help ensure you receive more substantive and useful feedback to advance in your career.

%%%%%%%
\subsection{Chapter Quiz}
\textbf{Q1.} Olivia is in a meeting with several executives discussing downsizing the company. The conversation is getting pretty heated. Which strategy would prevent Olivia from managing her emotions?\\
a) Go with the flow.\\
b) Breathe. \\  
c) Define the situation. \\  
d) Take a break. \\
\textbf{Ans: a} \\
Staying in a heated meeting will cause more anger and stress, which will have an overall negative effect on your ability to manage emotions.\\
\textbf{Q2.} Amelia is like many other people who have a hard time accepting praise. She feels unqualified, incapable, and not intelligent enough. This is known as \_\_\_\_\_.\\
a) set-up-to-fail syndrome \\ 
b) shadow syndrome \\  
c) imposter syndrome \\
\textbf{Ans: c} \\
This syndrome makes you feel like a fraud, and you fear that others may think you are incompetent. Thus, you believe your success is only pure luck.

%%%%%%%%%%%%%%%%%%%% %Communication strategies for women    %%%%%%%%%%%%%%%%%%%%%%%%%%%%
\section{ Communication strategies for women}

%%%%%%%
\subsection{Women and body language}
\textbf{Approach:} Men prefer being approached from the side, while women feel safer when approached from the front.\\
\textbf{Touch:} Be cautious with touch, especially as a supervisor, to avoid conveying unintended dominance.\\
\textbf{Timing:} Keep handshakes brief and maintain appropriate eye contact to avoid discomfort.\\
\textbf{Confidence:} Display confidence through your body language to help others feel at ease.\\
\textbf{Consistency:} Ensure your nonverbal cues match your verbal message to build trust and authenticity.

%%%%%%%
\subsection{Stop apologizing}
\textbf{Apologize only when at fault:}  Avoid saying "sorry" unless you have made a mistake or done something wrong.\\
\textbf{Use empathetic language without apologizing:} Instead of apologizing for circumstances outside your control, express empathy with phrases like "Gosh, I hate you had that experience."\\
\textbf{Avoid starting sentences with "I'm sorry":}  If you didn't hear a comment, say "Can you repeat that?" instead of "I'm sorry, can you repeat that?"\\
\textbf{Disagree without apologizing:}  State your disagreement confidently without prefacing it with an apology.\\
\textbf{Express gratitude instead of apologizing for minor errors:} Use phrases like "Thank you for bringing this to our attention" or "Great catch, I will make adjustments." instead of apologizing.\\
\textbf{AI tools to avoid weak language:} You can use AI tools to avoid weak language in e-mail communication.

%%%%%%%
\subsubsection{The double bind: Being assertive and likable}
\textbf{Double Bind Challenge:}  Women often face a dilemma where being assertive can make them seem competent but less likable, while being less assertive can make them seem likable but less competent.\\
\textbf{Signaling Strategies:}  Before asserting strong ideas, give a warning or explain your motivation to soften the impact and reduce the likelihood of being perceived as abrasive.\\
\textbf{Personal Choice:}  Decide whether to prioritize being liked or being effective based on the situation. You can choose to assert yourself when it matters most and be less assertive in other situations.

%%%%%%%
\subsection{Strengths women bring to negotioation}
\textbf{Creative Options:}  Women often identify more creative solutions and see the broader context of a situation, which can lead to innovative outcomes in negotiations.\\
\textbf{Relationship Focus:}  Women pay close attention to relationships and emotions, balancing tangible and emotional aspects of agreements, which can lead to long-term benefits and ethical sensitivity.\\
\textbf{Advocacy for Others:}  Women excel in negotiations when advocating for others, often outperforming men, especially when negotiating on behalf of subordinates. This strong sense of responsibility can create more opportunities for their team.\\

These strengths can make women highly effective negotiators, focusing on long-term impacts rather than just immediate gains.

%%%%%%%
\subsection{Communicating with confidence}
\textbf{Avoid weak language:}  Use clear and direct language. Replace phrases like "I think" or "I feel" with stronger statements like "I expect" or "My analysis shows".\\
Eg: "That report was so bad" can be stated as "That report lacks focus." Always be specific.\\
\textbf{Eliminate qualifiers:}  Avoid using qualifiers like "just" or preambles like "I'm not the expert but" as they can undermine your credibility.\\
\textbf{Avoid intensifiers:}  Words like "so" and "really" can be unnecessary fillers. Be specific and clear in your statements.\\
\textbf{Give clear directives:}  Be direct when assigning tasks or making requests to ensure clarity and effectiveness.\\
\textbf{Accept praise:} When receiving compliments, accept them graciously to boost your confidence and credibility.

%%%%%%%
\subsection{Chapter quiz}
\textbf{Q1.} Don is a very social guy, but when he meets people, he shakes their hand while he converses with them. Then he stares them down. Which tip would you recommend Don use in his nonverbal communication?\\
a) Show that you are confident. \\
b) Think about how you physically approach a conversation.\\
c) Control your timing. \\
d) Be careful with touch. \\
\textbf{Ans. C} \\
A handshake or eye contact lasting more than two seconds can make others feel uncomfortable.\\
\textbf{Q2.} Lorrie is a smart employee who always says she is sorry to the team every time she disagrees with a process. Which tip would you recommend Lorrie use in her communication?\\
a) Avoid starting sentences with "I'm sorry." \\
b) Apologize only when you are at fault.\\
c) Strive for continuous improvement.\\
d) Consider expressing gratitude for the correction instead of apologizing. \\
\textbf{Ans. C} \\
Lorrie should not feel badly for having an opinion. She should simply state that she disagrees and then explain why.

%%%%%%%%%%%%%%%%%%%% 	Championing women's leadership	 %%%%%%%%%%%%%%%%%%%%%%%%%%

%%%%%%%
\section{Championing women's leadership}
\subsection{Encouraging male allies}
\textbf{Educate without assigning blame:}  Help male colleagues understand unconscious bias with broader data and company-specific facts.\\
\textbf{Avoid cynicism:}  Stay positive and supportive towards male allies, even if their actions seem cliché.\\
\textbf{Advice for male allies:}
    \begin{itemize}
        \item \textbf{Listen up:} Actively listen to understand.
        \item \textbf{Speak up:} Address inappropriate behavior or comments.
        \item \textbf{Show up:} Participate in initiatives and groups that support gender equality.
    \end{itemize}

%%%%%%%
\subsection{How teams can leverage both masculine and feminine leadership}
\textbf{Utilize Agendas Wisely:}  Distribute agendas in advance and set time limits for each item to ensure efficient and thorough discussions.\\
\textbf{Make Debate the Norm:}  Encourage conflict and designate a devil's advocate to challenge ideas, leading to better solutions.\\
\textbf{Ensure Equal Numbers:}  Maintain a balanced number of men and women to value and support both perspectives.\\
\textbf{Rotate Meeting Leadership:}  Give everyone a chance to lead meetings, signaling that all members and perspectives are important.

%%%%%%%
\subsection{Cross-gender mentoring and sponsorship}
\textbf{Acknowledge discomfort:}  Openly discuss the challenges of cross-gender mentoring relationships to foster honest communication.\\
\textbf{Meet openly:}  Schedule meetings in public places and during regular work hours to reduce suspicion.\\
\textbf{Be public with motives:}  Clearly state the reasons for the mentoring relationship and the goals to be achieved.\\
\textbf{Know gender tendencies:}  Understand common gender behaviors and expectations to manage the mentoring relationship effectively.

%%%%%%%
\subsection{Chapter quiz}
\textbf{Q1.}  Lakshmi is in charge of reaching gender equality in the workplace. She knows a powerful way to do this is partnering with allies, especially men. Which action would you recommend Lakshmi use?\\
a)  Avoid cynicism.\\
b)  Use anecdotal examples.\\
c)  Avoid sarcasm. \\
d)  Educate without blaming.\\
\textbf{Ans. D} \\
It improve the understanding of unconscious bias and the role it plays in decisions with hiring, developing, and promotions in the company.\\

\end{document}
