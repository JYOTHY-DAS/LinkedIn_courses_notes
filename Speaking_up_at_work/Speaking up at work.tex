\documentclass[12pt]{article}
\usepackage[utf8]{inputenc}
\usepackage{graphicx}
\usepackage[colorlinks=true, linkcolor=black]{hyperref}
\usepackage{float} 
\usepackage{caption}
\usepackage{subcaption}
\usepackage{multicol}
\usepackage{comment}
\usepackage{booktabs} % For better-looking tables 
\usepackage{array} % For more control over table column formatting
\usepackage{colortbl} %Coloring table
\usepackage{xcolor}%Support hex values of color

\begin{document}

%%%%%%%%%%%%%%%%%%%%%%                Front page               %%%%%%%%%%%%%%%%%%%%%%%

\title{\textbf{Speaking up at work}}
\author{ Jessica Chen}
\maketitle

%%%%%%%%%%%%%%%%%      Table of Contents, List of Figures, and List of Tables       %%%%%%%%%%%%%%
\newpage
\tableofcontents
\newpage
%\listoffigures
%\newpage
%\listoftables

%%%%%%%%%%%%%%%%%%%%%%%%%%%%     Discovering  your mental barriers       %%%%%%%%%%%%%%%%%%%%%%%%%%%%%
\newpage
\section{Discovering  your mental barriers}
\begin{itemize}
\item{Mental barriers are obstacles that hinder effective communication, often stemming from past experiences.}
\item{These barriers can be formed by negative feedback or criticism received in the past, which can impact your confidence in speaking up.}
\item{To become a better communicator, it's crucial to identify these mental barriers and understand how they can be used to help rather than hinder your communication efforts.}
\end{itemize}
%%%%%%%
\subsection{How childhood and cultural experiences impact communication}
\textbf{Influence of upbringing:} Childhood teachings, such as being obedient and not speaking over elders, can hinder communication in the business world.\\
\textbf{Cultural norms:} Cultural experiences, like frequent comparisons or lack of positive reinforcement, shape how we express ourselves.\\
\textbf{Adapting to the workplace:} It's important to separate these past influences from current professional needs to effectively communicate and showcase your ideas at work.

%%%%%%%
\subsection{How work experiences impact communication}
\textbf{Past negative experiences:}  Negative experiences, such as nerve-wracking presentations or awkward meetings, can create mental barriers to speaking up.\\
\textbf{Overcoming these barriers:}  It's important to separate yourself from past experiences and focus on the present opportunity.\\
\textbf{Preparation strategies:}  Practice speaking up, maintain high energy, and seek feedback from colleagues to build confidence and improve communication skills.

%%%%%%%
\subsection{Chapter Quiz}
\textbf{Q1.} What's an example of a bad work experience that can stay with you?\\
a) You gave a bad presentation\\
 b) Everyone stayed silent after you shared your idea \\
 c) You felt people looked at you funny when you spoke up\\
 d) All the above\\
\textbf{Ans: D}\\
\textbf{Q2.} What's an example of a childhood experience that can affect your working behavior?\\
a) When you went on your favorite childhood vacation.\\
b) When your parents said it's better to stay quiet than to share your thoughts.\\
c) when you got hurt on the playground in elementary school.\\
d) when you had a disagreement with your childhood best friend.\\
\textbf{Ans: B}\\
\textbf{Q3.}Where do mental barriers come from?\\
a) Cultural experiences \\
b) Childhood experiences \\
c) Work experiences\\
d) All the above\\
\textbf{Ans: D}\\


%%%%%%%%%%%%%%%%%%%%%                 Steps to speak up               %%%%%%%%%%%%%%%%%%%
\section{ Steps to speak up}

%%%%%%%
\subsection{Make your voice an asset, not a liability}
\textbf{Strategic Communication Path (SCP):} Focus on three elements to use your voice effectively:\\
\begin{itemize}
\item{Environment: Create the desired environment for the conversation (e.g., warm or serious).}
\item{Role: Identify the role you want to play and the value you add to the conversation.}
\item{Energy: Adjust your energy level to shape the type of conversation (high for engagement, medium for balanced discussions, low for serious topics).}\\
\end{itemize}
These elements help you approach communication strategically and make your voice an asset.

%%%%%%%
\subsection{Prepare every time you speak}
\textbf{Simplify your message:} Keep your message simple and concise to ensure it is easily understood.\\
\textbf{Lead with interest:}  Start with the most interesting part of your message to capture attention.\\
\textbf{End with a takeaway:}  Conclude with the main point you want others to remember.\\
These steps will help you communicate more effectively and confidently.

%%%%%%%
\subsection{Speaking up in a meeting}
\textbf{Mental preparation:} Challenge yourself to speak up early in the meeting to build confidence and avoid self-doubt.\\
\textbf{Value add:} Identify your unique insight or knowledge that others may not have and share it.\\
\textbf{Pose questions:} If unsure about being direct, frame your thoughts as questions to contribute without being forceful.

%%%%%%%
\subsection{Speaking to your boss or superior}
\textbf{Remember your boss is human:}  Recognize that your boss has likes, quirks, and dislikes, which can help ease anxiety.\\
\textbf{Offer solutions, not just problems:}  Demonstrating proactive problem-solving and critical thinking skills is crucial.\\
\textbf{Avoid surprises:}  Keep your boss informed to prevent them from learning important information from others.\\
\textbf{Use their name:}  Saying your boss's name during conversations shows comfort and confidence.\\
These points can help you communicate more effectively and confidently with your boss.

%%%%%%%
\subsection{Chapter Quiz}
\textbf{Q1.} What is NOT a part of mentally preparing yourself to speak up in a meeting?\\
 a) Telling yourself you need to speak up at least once.\\
 b) Be OK not having people respond to your idea.\\
 c) Asking someone else to speak for you\\
 d) Creating bullet points\\
\textbf{Ans: C}\\
\textbf{Q2.} You always want to lead your conversations with the most important part because \_\_\_\_\_.\\
a) It is the easiest thing to do.\\
b) It has more impact if you build up to it gradually.\\
c) It captures the listener's attention\\
\textbf{Ans: C}\\
It's much easier to retain your audience's' attention when you can capture it from the very beginning. It's much more difficult to earn it back if they don't feel it's interesting, inspiring, motivational, etc from the very beginning.\\
\textbf{Q3.} What are the elements that make up the Strategic Communications Path?\\
a) Your role\\
b) Your environment \\
c) All of these answers\\
d) Your energy\\
\textbf{Ans: C}\\
\textbf{Q4} Which is NOT part of the four power pillars?\\
a) Saying their name \\
b) Communicating on the fly\\
c) Understanding your boss' communication style\\
d) No surprises\\
\textbf{Ans: B}\\

%%%%%%%%%%%%%%%%%%%%%                Confidently speaking               %%%%%%%%%%%%%%%%%%%

%%%%%%%
\subsection{Understanding your audience before speaking up}
\textbf{Tailor your communication:} Effective communication requires understanding what your audience cares about and tailoring your message accordingly.\\
\textbf{Three dynamics to consider:} 
\begin{itemize}
\item \textbf{Speaking to superiors:}  Build credibility quickly by keeping your information concise and high-level.
\item \textbf{Speaking to peers:}  Establish trust by being vulnerable and sharing your thoughts or seeking their input.
\item \textbf{Speaking to juniors:}  Show that you care about them by sharing the vision of the project to motivate and inspire them.
\end{itemize}

%%%%%%%
\subsection{Sample conversation starters}
\textbf{Ask smart questions:}  Avoid questions that lead to one-word answers. Instead, ask open-ended questions like "What brought you to this event?" or "How has that project you've been working on?"\\
\textbf{Use power words:}  Incorporate words that evoke emotional responses, such as "excited"(Eg: I was really excited about this.) or "frustrated,"(Eg: That project got me really frustrated)  to make your conversation more engaging.\\
\textbf{Build rapport through small talk:} Engaging in small talk helps develop rapport. Share a bit about yourself and show personality during the conversation.

%%%%%%%
\subsection{Chapter Quiz}
\textbf{Q1.} Which choice is NOT a strong, smart question?\\
 a) What brought you to this event?\\
 b) Did you eat lunch yet? \\
 c) How is your project going?\\
 d) What do you have planned over the weekend?\\
\textbf{Ans: B}\\
\textbf{Q2.} When you speak to those more junior than you, one of the most effective ways to motivate them is to show that you \_\_\_\_\_.\\
a) have been at the company for many years.\\
b) have an impressive background.\\
c) know how to engage in small talk.\\
d) care\\
\textbf{Ans: D}\\


%%%%%%%%%%%%%%%%%%%%%                Conclusion              %%%%%%%%%%%%%%%%%%%
\section{Conclusion}

%%%%%%%
\subsection{Speak up with confidence and clarity}
\textbf{Identify and overcome mental barriers:}  Reflect on past experiences that might have held you back and let them go to move forward.\\
\textbf{Understand your audience:}  Tailor your communication by considering what your audience cares about and their dynamic with you.\\
\textbf{Practice consistently:} Regular practice is essential to build confidence in your communication skills.\\
\textbf{Engage in small talk:} Use smart questions and power words to make your conversations more effective and engaging.

\end{document}
