\documentclass[12pt]{article}
\usepackage[utf8]{inputenc}
\usepackage{graphicx}
\usepackage[colorlinks=true, linkcolor=black]{hyperref}
\usepackage{float} 
\usepackage{caption}
\usepackage{subcaption}
\usepackage{multicol}
\usepackage{comment}
\usepackage{booktabs} % For better-looking tables 
\usepackage{array} % For more control over table column formatting
\usepackage{colortbl} %Coloring table
\usepackage{xcolor}%Support hex values of color

\begin{document}

%%%%%%%%%%%%%%%%%%%%%%                Front page               %%%%%%%%%%%%%%%%%%%%%%%

\title{\textbf{Tips to work with difficult people}}
\author{Emily Anhalt}
\maketitle

%%%%%%%%%%%%%%%%%      Table of Contents, List of Figures, and List of Tables       %%%%%%%%%%%%%%
\newpage
\tableofcontents
\newpage
%\listoffigures
%\newpage
%\listoftables

%%%%%%%%%%%%%%%%%%%%%%%%%%%%     Discovering  your mental barriers       %%%%%%%%%%%%%%%%%%%%%%%%%%%%%
\newpage
\section{Self awareness and mindset shifts}

%%%%%%%
\subsection{Reframe 'difficult people' to 'difficult behavior'}
\textbf{Reframe Perspectives:}  Focus on addressing difficult behaviors rather than labeling individuals as difficult.\\
\textbf{Manage Personal Triggers:}  Track and manage your own triggers to respond thoughtfully instead of reacting immediately.\\
\textbf{Utilize Empathy:}  Shift from judgment to curiosity to enhance understanding and connection.\\
\textbf{Build Trust:}  Proactively prepare for potential challenges to ensure smoother interactions.\\
\textbf{Set Boundaries:}  Create collaborative action plans for mutual growth and resolution while upholding boundaries.

%%%%%%%
\subsection{Track your triggers to go from reaction to response}
\textbf{Acknowledge Physical Sensations:}  Recognize the physical signs when you feel triggered, such as a faster heartbeat or clenched jaw.\\
\textbf{Connect Sensations to Triggers:}  Identify the behavior that caused the physical reaction, like feeling your heart race when receiving critical feedback.\\
\textbf{Create an If-Then Plan:}  Develop a plan to handle triggers thoughtfully, such as taking a breath and asking for time to think before responding to criticism.

%%%%%%%
\subsection{Reframe challenging interactions with a language switch}
\textbf{Language Reframe:}  Shift your mindset from "I have to" to "I get to" when dealing with difficult behaviors. This small change can transform challenges into opportunities for growth.\\
\textbf{Skill Development:}  Use challenging interactions as opportunities to practice and develop valuable skills, such as collaboration and leadership.\\
\textbf{Positive Mindset:}  Adopting this language switch helps put you in the right mindset to learn and grow from difficult situations.

%%%%%%%
\subsection{Filter for impact to decide what to address}
\textbf{Assess Impact:}  Determine if the difficult behavior is directly impacting your work. If it affects your deadlines, it's worth addressing.\\
\textbf{Identify Patterns:}  Check if the behavior is an ongoing problem or a one-off situation. Patterns are more important to address.\\
\textbf{Evaluate Influence:}  Consider if you have any influence over the behavior. If it's within your control, take action; if not, seek help from those who can influence it.

%%%%%%%
\subsection{Move from judgment to curiosity}
\textbf{Replace Judgment with Curiosity:}  Instead of making quick judgments about others' actions, use phrases like "I wonder" to explore the reasons behind their behavior.\\
\textbf{Use Empathetic Language:} Phrases like "Help me understand" and "I'm curious about" can open up dialogue and foster better communication.\\
\textbf{Encourage Collaboration:} Curiosity can disarm people and make them feel seen and understood, which promotes collaboration.

\section{Strategic communication tools}
%%%%%%%
\subsection{Build trust and relational capital before conflict happens}
\textbf{Start with Personal Check-Ins:}  Begin work meetings with a brief personal check-in to build rapport.\\
\textbf{Acknowledge Contributions:}  Regularly acknowledge and validate people's contributions to build trust over time.\\
\textbf{Follow Through on Commitments:}  Do what you say you will do, and if you can't, acknowledge and apologize.

%%%%%%%
\subsection{Set yourself up for success with reactive colleagues}
\textbf{Provide Processing Time:}  Give reactive colleagues time to process information by sending agendas in advance and breaking big tasks into smaller ones.\\
\textbf{Use Tentative Language:} Use phrases like "I'm thinking about suggesting X, but wanted your perspective first" to lower defenses.\\
\textbf{Opt-In for Tough Conversations:}  Allow colleagues to choose a suitable time for difficult discussions to ensure they are prepared and open.

%%%%%%%
\subsection{Take accountability for your actions}
\textbf{Own Your Part:}  Acknowledge your role in any conflict, no matter how small, to encourage others to do the same.\\
\textbf{Use Accountability Phrases:} Phrases like "I contributed to this problem by..." and "I could have handled this better by..." can help in taking responsibility.\\
\textbf{Create Space for Collaboration:}  Taking responsibility doesn't mean letting others off the hook; it creates space for collaborative repair and change.

%%%%%%%
\subsection{Go from 'me vs. you' to 'us vs. the problem}
\textbf{Shift Mindset:}  Reframe conflicts from "me vs. you" to "us vs. the problem" to foster collaboration.\\
\textbf{Common Goals:} Remind yourself that both parties are on the same team and share common goals.\\
\textbf{Collaborative Solutions:} Work together to find solutions that address the problem without wasting time or excluding important stakeholders.

%%%%%%%
\subsection{Address both party's needs}
\textbf{Use Both/And Approach:}  Start by saying, "Let's find a solution that works for both of us," and name both perspectives to ensure everyone's needs are addressed.\\
\textbf{Communicate Needs Clearly:}  State your need and acknowledge the other person's need, e.g., "I need a clear deadline, and I understand you need flexibility."\\
\textbf{Propose Solutions Together:}  Offer a possible solution and ask for their thoughts, aiming for a middle ground that works for both parties.

%%%%%%%
\subsection{Swap positions to exchange viewpoints}
\textbf{Position Swap Exercise:}  When in a disagreement, swap positions and argue the other person's side to understand their perspective better.\\
\textbf{Loosen Attachment to Being Right:} This exercise helps reduce the need to be right and opens up new points of view.\\
\textbf{Enhanced Understanding}: By seeing the other person's perspective, you might discover new insights that can lead to better decision-making.

%%%%%%%
\section{Boundaries and action plans}
\subsection{Build your network for ongoing support}
\textbf{Identify Key Support Roles:}  Build a support network including a sounding board, a problem solver, and an official support.\\
\textbf{Proactive Relationship Building:}  Establish these relationships before you need them by scheduling regular check-ins and offering support in return.\\
\textbf{Maintain Professional Boundaries:} Ensure you maintain professional boundaries while building and nurturing these relationships.

%%%%%%%
\subsection{Learn a formula to uphold boundaries}
\textbf{Name the Behavior:}  Clearly state the behavior without judgment, e.g., "I've noticed that you text my personal phone after work hours."\\
\textbf{Express Your Need:} Communicate your need clearly, e.g., "I need to maintain a healthy work-life balance."\\
\textbf{State Your Next Step:}  Outline what you will do to protect your need, e.g., "So I'll respond to after-hours messages the next business day."

%%%%%%%
\subsection{When to disengage from conflict professionally}
\textbf{Temporary Pauses:}  Sometimes it's necessary to take a temporary pause to let emotions settle.\\
\textbf{Professional Boundaries:}  If professional boundaries are being crossed or no progress is being made, it might be time to implement an exit strategy.\\
\textbf{Clear Communication:}  Maintain dignity for all parties, use clear communication, and involve your HR manager if needed.

\end{document}
