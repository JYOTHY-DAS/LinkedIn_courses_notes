\documentclass[12pt]{article}
\usepackage[utf8]{inputenc}
\usepackage{graphicx}
\usepackage[colorlinks=true, linkcolor=black]{hyperref}
\usepackage{float} 
\usepackage{caption}
\usepackage{subcaption}
\usepackage{multicol}
\usepackage{comment}
\usepackage{booktabs} % For better-looking tables 
\usepackage{array} % For more control over table column formatting
\usepackage{colortbl} %Coloring table
\usepackage{xcolor}%Support hex values of color

\begin{document}

%%%%%%%%%%%%%%%%%%%%%%                Front page               %%%%%%%%%%%%%%%%%%%%%%%

\title{\textbf{A guide to understanding financial statements}}
\author{Josh Aharonoff}
\maketitle

%%%%%%%%%%%%%%%%%      Table of Contents, List of Figures, and List of Tables       %%%%%%%%%%%%%%
\newpage
\tableofcontents
\newpage
%\listoffigures
%\newpage
%\listoftables

%%%%%%%%%%%%%%%%%%%%%%%%%%%%     Discovering  your mental barriers       %%%%%%%%%%%%%%%%%%%%%%%%%%%%%
\newpage
\section{Introduction}

%%%%%%%
\subsection{Why you need to understand financial statements}
\textbf{Purpose and Structure:}  Learn the purpose and structure of financial statements, including profit and loss, balance sheet, and cash flow statement.\\
\textbf{Key Components:} Identify and describe key components like revenue, cost of goods sold, operating expenses, and net income in a profit and loss statement.\\
\textbf{Accounting Methods:}  Differentiate between cash basis and accrual basis accounting and understand their implications.\\
\textbf{Financial Health:}  Analyze the relationship between assets, liabilities, and owners' equity on a balance sheet, and predict the impact of financial transactions on business sustainability.\\
\textbf{Standards and Principles:}  Evaluate the differences between GAAP( Generally Accepted Accounting Principles) and IFRS(International Financial Reporting Standards) and understand the preparation methods of cash flow statements.

%%%%%%%
\subsection{What are financial statements?}
\textbf{Definition:} Financial statements are like lab results for your company's financial health, showing income, assets, and liabilities.\\
\textbf{Types of Statements:} The three main financial statements are:\\
\begin{itemize}
\item Profit and Loss Statement: Details income and expenses, showing profitability.
\item Balance Sheet: Shows assets, liabilities, and owners' equity at a specific point in time.
\item Statement of Cash Flows: Tracks cash movements through operating, investing, and financing activities.
\end{itemize}
\textbf{Accounting Methods:}  The video introduces cash basis and accrual basis accounting, which will be explored further.

%%%%%%%
\subsection{Cash vs. accrual}
\textbf{Cash Basis:}
\begin{itemize}
\item Records transactions when cash is received or paid.
\item Simpler and provides a real-time view of cash flow.
\item Not compliant with GAAP, and may not fully reflect business activities.
\end{itemize}
\textbf{Accrual Basis:}
\begin{itemize}
\item Records transactions when income is earned and expenses are incurred.
\item More complex but provides a more accurate picture of financial health.
\item Compliant with GAAP and used by larger businesses with outside investors.
\end{itemize}


%%%%%%%
\subsection{GAAP vs. IFRS}
\textbf{Definitions:}
\begin{itemize}
\item GAAP (Generally Accepted Accounting Principles): Common in the United States, especially for public companies.
\item IFRS (International Financial Reporting Standards): A unified standard used globally.
\end{itemize}

\textbf{Usage:}
\begin{itemize}
\item GAAP is developed by the Financial Accounting Standards Board (FASB) and is mandatory for U.S. public companies.
\item IFRS is developed by the International Accounting Standards Board (IASB) and is used internationally.
\end{itemize}

\textbf{Accounting Basis:}
\begin{itemize}
\item GAAP does not accept the cash basis of accounting and requires the accrual basis.
\item Both standards provide a rule book for recognizing transactions, ensuring reliability in financial statements.
\end{itemize}


%%%%%%%
\subsection{Chapter Quiz}
\textbf{Q1.} Why do businesses prepare financial statements?\\
a) to reduce the amount of taxes paid\\
 b) to provide a summary of financial performance and position to stakeholders\\
 c) to fulfill the legal obligation of filing with the IRS\\
 d) to list all of the company's expenses for the year\\
\textbf{Ans: B}\\
Financial statements inform investors, management, and other stakeholders about the company’s financial health.\\
\textbf{Q2.} How does the accrual basis of accounting differ from the cash basis when recognizing revenue?\\
a) In Accrual accounting, revenue is recognized when it is earned, regardless of when the cash is received.\\
b) Revenue is recognized at the beginning of the fiscal year, regardless of when it is earned or paid.\\
c) Revenue is recognized only when cash is received, even if the service or product hasn't been provided yet.\\
d) Revenue is recognized only when the product is delivered and the customer provides feedback.\\
\textbf{Ans: A}\\
In accrual accounting, revenue is recognized when earned, even if the cash is received later.\\


\end{document}
